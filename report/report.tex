% !TeX program = pdfLaTeX
\documentclass[12pt]{article}
\usepackage{amsmath}
\usepackage{graphicx,psfrag,epsf}
\usepackage{enumerate}
\usepackage{natbib}
\usepackage{textcomp}
\usepackage[hyphens]{url} % not crucial - just used below for the URL
\usepackage{hyperref}
\providecommand{\tightlist}{%
  \setlength{\itemsep}{0pt}\setlength{\parskip}{0pt}}

%\pdfminorversion=4
% NOTE: To produce blinded version, replace "0" with "1" below.
\newcommand{\blind}{0}

% DON'T change margins - should be 1 inch all around.
\addtolength{\oddsidemargin}{-.5in}%
\addtolength{\evensidemargin}{-.5in}%
\addtolength{\textwidth}{1in}%
\addtolength{\textheight}{1.3in}%
\addtolength{\topmargin}{-.8in}%

%% load any required packages here


\usepackage{color}
\usepackage{fancyvrb}
\newcommand{\VerbBar}{|}
\newcommand{\VERB}{\Verb[commandchars=\\\{\}]}
\DefineVerbatimEnvironment{Highlighting}{Verbatim}{commandchars=\\\{\}}
% Add ',fontsize=\small' for more characters per line
\usepackage{framed}
\definecolor{shadecolor}{RGB}{248,248,248}
\newenvironment{Shaded}{\begin{snugshade}}{\end{snugshade}}
\newcommand{\AlertTok}[1]{\textcolor[rgb]{0.94,0.16,0.16}{#1}}
\newcommand{\AnnotationTok}[1]{\textcolor[rgb]{0.56,0.35,0.01}{\textbf{\textit{#1}}}}
\newcommand{\AttributeTok}[1]{\textcolor[rgb]{0.77,0.63,0.00}{#1}}
\newcommand{\BaseNTok}[1]{\textcolor[rgb]{0.00,0.00,0.81}{#1}}
\newcommand{\BuiltInTok}[1]{#1}
\newcommand{\CharTok}[1]{\textcolor[rgb]{0.31,0.60,0.02}{#1}}
\newcommand{\CommentTok}[1]{\textcolor[rgb]{0.56,0.35,0.01}{\textit{#1}}}
\newcommand{\CommentVarTok}[1]{\textcolor[rgb]{0.56,0.35,0.01}{\textbf{\textit{#1}}}}
\newcommand{\ConstantTok}[1]{\textcolor[rgb]{0.00,0.00,0.00}{#1}}
\newcommand{\ControlFlowTok}[1]{\textcolor[rgb]{0.13,0.29,0.53}{\textbf{#1}}}
\newcommand{\DataTypeTok}[1]{\textcolor[rgb]{0.13,0.29,0.53}{#1}}
\newcommand{\DecValTok}[1]{\textcolor[rgb]{0.00,0.00,0.81}{#1}}
\newcommand{\DocumentationTok}[1]{\textcolor[rgb]{0.56,0.35,0.01}{\textbf{\textit{#1}}}}
\newcommand{\ErrorTok}[1]{\textcolor[rgb]{0.64,0.00,0.00}{\textbf{#1}}}
\newcommand{\ExtensionTok}[1]{#1}
\newcommand{\FloatTok}[1]{\textcolor[rgb]{0.00,0.00,0.81}{#1}}
\newcommand{\FunctionTok}[1]{\textcolor[rgb]{0.00,0.00,0.00}{#1}}
\newcommand{\ImportTok}[1]{#1}
\newcommand{\InformationTok}[1]{\textcolor[rgb]{0.56,0.35,0.01}{\textbf{\textit{#1}}}}
\newcommand{\KeywordTok}[1]{\textcolor[rgb]{0.13,0.29,0.53}{\textbf{#1}}}
\newcommand{\NormalTok}[1]{#1}
\newcommand{\OperatorTok}[1]{\textcolor[rgb]{0.81,0.36,0.00}{\textbf{#1}}}
\newcommand{\OtherTok}[1]{\textcolor[rgb]{0.56,0.35,0.01}{#1}}
\newcommand{\PreprocessorTok}[1]{\textcolor[rgb]{0.56,0.35,0.01}{\textit{#1}}}
\newcommand{\RegionMarkerTok}[1]{#1}
\newcommand{\SpecialCharTok}[1]{\textcolor[rgb]{0.00,0.00,0.00}{#1}}
\newcommand{\SpecialStringTok}[1]{\textcolor[rgb]{0.31,0.60,0.02}{#1}}
\newcommand{\StringTok}[1]{\textcolor[rgb]{0.31,0.60,0.02}{#1}}
\newcommand{\VariableTok}[1]{\textcolor[rgb]{0.00,0.00,0.00}{#1}}
\newcommand{\VerbatimStringTok}[1]{\textcolor[rgb]{0.31,0.60,0.02}{#1}}
\newcommand{\WarningTok}[1]{\textcolor[rgb]{0.56,0.35,0.01}{\textbf{\textit{#1}}}}

% Pandoc citation processing

\usepackage{booktabs}
\usepackage{longtable}
\usepackage{array}
\usepackage{multirow}
\usepackage{wrapfig}
\usepackage{float}
\usepackage{colortbl}
\usepackage{pdflscape}
\usepackage{tabu}
\usepackage{threeparttable}
\usepackage{threeparttablex}
\usepackage[normalem]{ulem}
\usepackage{makecell}
\usepackage{xcolor}

\begin{document}


\def\spacingset#1{\renewcommand{\baselinestretch}%
{#1}\small\normalsize} \spacingset{1}


%%%%%%%%%%%%%%%%%%%%%%%%%%%%%%%%%%%%%%%%%%%%%%%%%%%%%%%%%%%%%%%%%%%%%%%%%%%%%%

\if0\blind
{
  \title{\bf Detecting Pneumonia within CT Scans Using Convolutional Neural Networks}

  \author{
        Graham Chickering \\
    \\
      }
  \maketitle
} \fi

\if1\blind
{
  \bigskip
  \bigskip
  \bigskip
  \begin{center}
    {\LARGE\bf Detecting Pneumonia within CT Scans Using Convolutional Neural Networks}
  \end{center}
  \medskip
} \fi

\bigskip
\begin{abstract}
Convolutional Networks are a specific type of Neural Networks that have
shown to be particularly effective at being able to identify distinct
objects within images. This technique can be used to identify different
sorts of condition, such as pneumonia, within medical images as well as
detect other sorts of tumors or cancers. In theory this type of network
is designed based on how the human brain works and the idea that
multiple levels of neurons are connected together in order to detect and
identify images. In practice though, running and training these types of
neural networks can be very computationally expensive and require large
amounts of memory and processing capabilities if working with a very
large dataset. Especially when working in R which has limited memory
capabilities, trying to run and train this type of model can be very
slow and ineffective. Thanks to developments in cloud computing such as
Google Cloud Storage and Tensorflow being able to store and run these
types of models within R can become much faster and more efficient when
trying to analyze large amounts of data. While there is more work to be
done, this project shows how to create an infrastructure that
efficiently stores data and then train convolutional neural networks
when working with the R Studio Environment.
\end{abstract}

\noindent%
{\it Keywords:} Convolutional Neural Networks, Google Cloud Storage, Tensorflow, Keras
\vfill

\newpage
\spacingset{1.45} % DON'T change the spacing!

\begin{figure}

{\centering \includegraphics[width=0.75\linewidth,height=0.25\textheight]{images/big-data-healthcare} 

}

\caption{Big Data in Healthcare}\label{fig:sample-fig1}
\end{figure}

\hypertarget{introduction}{%
\section{Introduction}\label{introduction}}

~~~~~It has been estimated that roughly 80\% of health care data is
unstructured data, which can come in the form of videos, sensor data,
images, or text. Although hospitals and researchers used to have a hard
time extracting insights from this type of data, with the recent
advances that have been made in data science and handling big data, this
has created new application areas within the health care industry in
sectors such as genomics, drug trials, predicting patient health, and
medical imagery (See Figure 1 \citet{NEJM}). Medical imaging research in
particular has made significant progress recently with researchers being
able to use different machine learning algorithms to detect different
types of lesions and cancers from CT and other types of scans. In
particular the advancements of convolutional neural networks to identify
whether or not someone has pneumonia has become extremely promising and
can be used to potentially help doctors identify whether or not someone
has pneumonia that they might have missed, and reduce the amount of
hours and amount of expertise required to view CT scans.

When working with medical imagery data sets one of the first problems
someone may run into is how to process and handle these large data sets.
When trying to perform analysis on small and medium sized data sets
within R, one rarely run into complications that would be attributed to
how R is loading and dealing with the data itself. But what happens when
one moves from the world of medium sized data to the world of Big Data
and large data sets? While most of the time one can load data into the R
Studio environment without issues and without having to worry about
whether the entirety of our data can even be loaded in, one may begin to
run into complications the larger the data set becomes. If one ends up
crossing into the threshold where R can no longer store all the data in
an effective way, there are multiple potential solutions in the forms of
choosing random subsets of the data, buying a computer with larger
memory, or use parallelization and using multiple clusters to perform
the analysis. It is this solution of choosing random subsets of data,
using the Keras and Tensorflow R packages, that will allow me to work
with and convert large files of image data into a form that models can
be trained on them.

On top of the issue of trying to run analysis on large data sets in R
itself, is the issue of how to best store and load the original
information and data. Often data sets are small enough that they can be
stored on your local computer in a folder that is then uploaded into the
R Studio Environment itself, but what should one do as the size of the
data set substantially increases and one no longer wants to store large
data sets directly on their machine. One solution to this problem is to
take advantage of a cloud computing service and store the data directly
in the cloud, freeing up space and memory on your personal computer. By
storing the data on a cloud computing service, this becomes especially
useful when a project begins to get scaled up whether that is through
adding new members to work on the project or when more and more data
gets added to the project. In this project I will take advantage of
Google Cloud Storage to store my data, saving space on my local machine.

By combining Google Cloud Storage with Keras and Tensorflow, this will
allow me to utilize a large data medcical imagery data set that is made
up of CT scans. This data set will then be used to train and create
convolutional neural networks that will try to identify whether or not
someone has pneumonia from the images.

This project will allow me to answer the questions of what is the best
way to store large data sets and perform computationally expensive
analysis on those data sets? How does one handle and process images so
that analysis can be performed on them, and how effective are
convolutional neural networks at identifying pneumonia within CT scans?

\hypertarget{google-cloud-storage}{%
\section{Google Cloud Storage}\label{google-cloud-storage}}

~~~~~When trying to work with Big Data, one of the first questions one
has to answer is what is the best way to store and access this data.
While smaller files can be stored directly on your computer and
eventually loaded into R Studio to perform analysis on, when data moves
into the gigabyte, terabyte, or even petabyte range one may not not want
to store this data directly on their machine and use of large chunks of
their limited memory that is available. With the advancement of cloud
service solutions in recent years though, one can now use a platform
such as Google Cloud Platform, Amazon Web Services, or Microsoft Azure,
to store large amounts of data directly on these platforms and take
advantage of these companies large data warehouses for a small cost.
This can allow one to free up space and memory on their own personal
machine and access the data directly from these servers whenever they
desire.

\begin{figure}

{\centering \includegraphics[width=0.75\linewidth,height=0.25\textheight]{images/cloud_storage} 

}

\caption{Google Cloud Storage Platform}\label{fig:sample-fig2}
\end{figure}

For this project, I am going to focus on how to setup and store medical
imagery in Google Cloud Storage and learn how this can be connected to R
Studio so that I can then perform analysis on these images. The data set
I will be working with is a labeled Chest X-Ray images (\citet{Medical})
which will be used to detect and classify whether or not someone has
pneumonia. This data set is roughly 2 GB in size and contains CT scans
of patients who either have or dont have pneumonia. Due to Github having
maximum storage limit of 1 GB, I wanted to look into ways that would
allow me to work with a data set of this size and not have to upload the
data directly into Github itself. One of the solutions for this was to
use Google Cloud Storage and take advantage of the free credits that
Google offers for new users using their service. By uploading and
storing the data set directly onto the Google platform, this meant I
could delete the data set off my computer for the time being and free up
memory space (See Figure 2 for an example of the Google Cloud Platform
Console).

In order to actually work with this data in R Studio though, a
connection between R and Google Cloud was required to be setup. By
utilizing the \emph{googleCloudStorageR} package, I was able to setup a
connection that allowed me to download the data from my Cloud Storage
bucket and onto my desktop where it could then be read into R Studio
without having to store the files themselves within R Studio (See
Apendix for code on how to setup this connection). This allowed me to
not have to store the data in my actual Github repository but still be
able to perform analysis and build models on the image dataset.

\hypertarget{image-transformation-and-tensorflow}{%
\section{Image Transformation and
Tensorflow}\label{image-transformation-and-tensorflow}}

~~~~~Once the images are in a place where they could be loaded into R,
one needs to put the images into a format where R can actually perform
analysis on them. For images, this means making it so they all have the
same underlying which can require different image transformations and
adjustments such as cropping, brightness, contrasting, changing the
color scale, or resizing an image. This sort of data augmentation, when
performed on all the images in the data set can help create a more
consistent form between all the images. For the images in my medical
imagery data set,this primarily consisted of resizing the images on a
pixel by pixel basis so they were all the same size, and then converted
them to a grayscale color. This made the underlying structure of the
images the same for all the images in the data set. See the Figure 3 for
an example between a patient with and without pneumonia after
adjustments were made.

\begin{figure}
\includegraphics[width=0.49\linewidth,height=0.25\textheight]{images/pneumonia} \includegraphics[width=0.49\linewidth,height=0.25\textheight]{images/normal} \caption{Patient CT Scans. At left, patient with pneumonia. At right, a patient without pnuemonia}\label{fig:sample-fig3}
\end{figure}

After getting all the images into the same structure, I needed to
convert them to a form where I could actually perform analysis on the
images. In order to do this I relied on the \emph{tensorflow} R
packages. Tensorflow is a ``scalable and multiplatform programming
interface for implementing and running machine learning algorithms''
\citet{PML}. Tensorflow allows execution on both CPU's and GPU's which
help speed up processing time on complicated algorithms. This package is
built around a ``computation graph composed as a set of nodes where each
node represents an operation that may have zero or more input or output.
A tensor is created as a symbolic handle to refer to the input and
output of these operations'' \citet{PML}. A tensor is best understood as
either a scalar, vector , matrices and so on, which all correspond to a
different rank tensor (See Figure 4). These tensors are created from the
values in the data you are working with and then are used to build and
create the complex models one wants to work with.

\begin{figure}

{\centering \includegraphics[width=0.75\linewidth,height=0.25\textheight]{images/tensors} 

}

\caption{Tensors in Tensorflow}\label{fig:sample-fig4}
\end{figure}

For the images in my data set, by using Tensorflow I was able to convert
the images into a set of tensors that represented the pixels of the
images themselves. Since the images had been resized to a 64x64 pixel
picture in grayscale coloring, this became a 64x64x1 tensor for each
image. These tensors were then combined with their appropriate label for
the type of image they were, either Normal or Pneumonia, creating a
nicely formatted data set where we could then start to create models to
best classify the data.

\hypertarget{keras-and-convolutional-neural-networks}{%
\section{Keras and Convolutional Neural
Networks}\label{keras-and-convolutional-neural-networks}}

~~~~~As we could see from the prior pictures of someone who had
pneumonia versus someone who did not have pneumonia, it can be very hard
to discern whether or not someone has pneumonia for a person without the
technical training and expertise to identify pictures of pneumonia from
looking at the CT scans. This ultimately requires doctors and those with
the expertise to spend more time looking at the scans, rather than
spending their already limited time directly helping patients. One way
to help doctors to get back to directly helping patients is by utilizing
machine learning to identify and detect different diseases or ailments
within CT scans. By training and creating models that are able to
discern between a normal or healthy scan versus someone who has a lesion
or has pneumonia, we can begin to use artificial intelligence to
alleviate some of the extraneous work that doctors are required to do.

\hypertarget{keras}{%
\subsection{Keras}\label{keras}}

~~~~~For this project, I was specifically focused on creating a model
that would would be able to discern between someone who has pneumonia
versus someone who does not have pneumonia. In order to do this I
utilized the \emph{keras} package in order to build a convolutional
neural network to be able to distinguish between the two different
diagnosis. Keras is a ``high-level neural network API that is built to
run off other libraries such as Tensorflow to provide a user-friendly
interface to building complex models'' \citet{PML}. Keras, when working
with Tensorflow, helps to provide the framework to begin building
complex neural network models. This package will allow me to be able to
not only train and build the model, but be able to test the model on
another subset of images and be able to begin to predict whether or not
someone has pneumonia from looking at a specific image. Specifically I
used the keras package to build out a convolutional neural network that
have been shown to work extremely well for image classification tasks.

\hypertarget{neural-networks}{%
\subsection{Neural Networks}\label{neural-networks}}

~~~~~Convolutional Neural Networks (CNN) at a high level are a form of
deep learning thattakes an image as an input and eventually classifies
it under a certain category. These types of networks are used heavily to
perform tasks such as facial recognition, object detection, and image
classification to just list a few. CNN's are a specific type of neural
network, which falls under the branch of machine learning called deep
learning. Neural networks are extremely popular today due to recent
advances in both the algorithms for the models and the computer
architecture which allows for much faster processing times of these
complex models.

\begin{figure}

{\centering \includegraphics[width=0.75\linewidth,height=0.25\textheight]{images/neural_net} 

}

\caption{Neural Network example}\label{fig:sample-fig5}
\end{figure}

The figure above shows an example of what the underlying structure of a
neural network looks like (See Figure 4 \citet{Investopedia}). The
neural network show in this example consists of one input layer, one
hidden layer with 4 hidden units, and then one output layer. The input
layer would be the exact data you are feeding into the model, so for
this picture this would mean there was two examples being fed into
network to train the model. Then we can see how each of the input layers
are connected to three of the gray nodes in the hidden layer. While this
model only has one hidden layer and the inputs are only connected to 3
of the hidden units, in other models one could create multiple hidden
layers, each with a different number of hidden units, and the number of
units that are connected together from layer to layer can vary as well.
After that we can see that there is one output layer for this example,
but other networks can contain varying numbers of output units as well.
Neural networks in general can contain differing number of hidden
layers, hidden units, input and output units and vary based on the data
they have and the problem they are trying to solve. In general data is
fed into the model, it is passed through the network using weights, and
then often trained using a technique called backwards propagation. This
technique is able to optimize the model by comparing the result found by
the model to the true results, and then working backwards through the
model to update the weights so that the model can better classify the
input data. Neural networks in general have been shown to be extremely
successful at a broad range of tasks such as natural language processing
to self-driving cars and everything inbetween.

\hypertarget{convolutional-neural-networks-cnn}{%
\subsection{Convolutional Neural Networks
(CNN)}\label{convolutional-neural-networks-cnn}}

~~~~~As mentioned earlier, CNN's a specific form of Neural Network that
is based heavily on how the visual cortext of the human brain works when
recogizing images, which is what allows it to perform extremely well on
image classification tasks. At a high level CNN's work by ``combining
the low level features in a layer-wise fashion to form high-level
features'' \citet{PML}. So rather than just looking at each pixel of an
image separately, this model works to combine pixels into distinct
features that can then be used to identify exact objects within the
images. In order for the model to actual identify distinct features it
relies on an idea called feature mapping that groups patches of pixels
together in the image and combines them into one feature in the new
feature map. This is based on the underlying assumption that in the
context of image data, ``nearby pixels are typically more relevant to
each other than pixels that are far away from each other'' \citet{PML}.

In order to actually perform this feature mapping though a CNN relies on
creating a series of different types of layers in the form of
convolution layers, subsampling layers, pooling layers, and dropout
layers.

\hypertarget{convolution-layers}{%
\subsubsection{Convolution layers}\label{convolution-layers}}

~~~~~Convolution layers is one of the first layers that begins
extracting features from the input images. This layer works by taking an
input matrix that represents the image and a filter matrix, of
potentially a different size, with a set of weights. These two matrices
are multiplied together to create a feature map that begins to identify
the low level features such as edges, blurriness, or sharpness of the
image.

\hypertarget{subsampling-and-pooling-layers}{%
\subsubsection{Subsampling and Pooling
Layers}\label{subsampling-and-pooling-layers}}

~~~~~Another type of key layer is the subsampling and pooling layers.
Usually the feature map that is created from the previous convolution
layer is then fed into a subsampling/pooling layer. These layers work by
combining small subsections of the feature map in order to simplify the
feature map. The advantages of this include leading to higher
computational efficiency by decreasing the size of the features and the
number of parameters that are required to learn, as well as introducing
local invariance that helps to generate feature that are more robust to
noise from the input images \citet{Mathworks}.

\hypertarget{dropout-layers}{%
\subsubsection{Dropout layers}\label{dropout-layers}}

~~~~~Dropout layers are then used to prevent over-fitting of the data
set. It can be very easy to create a CNN that gets over-trained but then
fails to perform well on the testing set of data. To prevent
over-fitting and make the model work well for a broader range of images
for general performance, dropout layers are introduced as a form of
regularization. Dropout is usually applied to hidden units of layers and
works by ``during the training phase of a neural network, a fraction of
the hidden units are randomly dropped at every iteration. This dropping
out of random units requires the remaining units to rescale to account
for the missing units which forces the network to learn a redundant
representation of the data'' \citet{PML}. This makes it so the model is
more generalizable and robust to changes in patterns in the data and
prevent over-fitting.

\hypertarget{challenges}{%
\subsubsection{Challenges}\label{challenges}}

~~~~~When working with CNN's, these different types of layers that the
network can be built off can all be included multiple times and in
different orders. For example one could create a network of just a
convolution layer and a dropout layer or someone could create a model
that is convolution layer, pooling layer, convolution layer, and then
two dropout layers. There is no set rules around what order or how many
different layers your network can have. On top of having unlimited
variations of the type of layers within the network, there is also
unlimited parameters that one can choose in terms of the size of the
feature map, the number of units to pool together in the pooling layer,
or the number of units to dropout in the dropout layer to just list a
few examples. Due to having so many input parameters that can be changed
and altered when it comes to building the model, finding the best
possible input parameters and underlying structure of the model can
become computationally expensive very quickly.

~~~~~On top of CNN's being very computationally expensive to run, being
able to interpret the outputs and inputs of individual layers of the
network is its own separate issue. Often these types of models are
referred to as ``black box'' models because while a human can understand
the input and outputs of the models, it is often hard to decipher and
understand what exactly the model is doing inside the box. While there
has been promising developments in this area such as Visualizing
Activation Layers and Occlusion Sensitivity, this still remains an area
of the field that a lot of future work can be done in \citet{Deep}.
While CNN's have been shown to do an incredible job at image
classification tasks, optimizing and understanding these types of models
is an area that can cause problems and issues.

\hypertarget{cnns-with-medical-images}{%
\subsection{CNN's with Medical Images}\label{cnns-with-medical-images}}

~~~~~For my project, I wanted to use Convolutional Neural Networks to
see how well this type of network could perform on the task of trying to
classify whether or not someone had pneumonia based just of a CT scan.
At this point I have already gotten my images into a form that allows
them to be passed as inputs to build a model, so the next step was for
me to begin training the model.

To begin training the model I started off by first splitting my data
into training, validation, and testing sets. The training set would be
used to train the model, the validation set would be used to tune the
parameters in the network, and then the test set would then be used to
assess the performance of the final model and see how generalizable the
results are when given separate input images.

After splitting up the data into their different sets I found that
overall the data set contained many more pictures of people than
pneumonia than people who did not have pneumonia. In fact the training
data contains 74\% picture of people with pneumonia and 26\% being
normal images.

\includegraphics[width=0.75\linewidth,height=0.25\textheight]{report_files/figure-latex/unnamed-chunk-4-1}

What this tells us is that if a random person were to guess that every
picture was a picture of pneumonia, that they would be correct 74\% of
the time. So when I build out my models, I will be looking for a model
that is able to achieve better than 74\% accuracy to say it is an
improvement over a baseline random guessing model.

\hypertarget{model-1}{%
\subsection{Model 1}\label{model-1}}

~~~~~Now that I had an understanding of what my training set looked
like, I wanted to try building my first convolutional neural network.
For my first model I decided to build a model that had one convolution
layer, one pooling layer, one flattening layer, a dropout layer that
removed half of the available units available in the previous layer
during each new training iteration, and then finally one dense layer.
The flattening layer works to replace all dimensions of the previous
tensors down to one dimension, which is the dimension size we want our
output layer to be. The dropout layer is used to make the model more
generalizable and requires the model to fit the units with only half the
units of the previous layer available during any iteration. The final
dense layer is used to create a layer of units, in this case one unit,
where every unit in this new layer is connected to every unit in the
previous layer, making it densely connected. See Figure 5 for what this
model looks likes. From looking at Figure 5 one can see that there are
13,121 trainable parameters. These are all the weights between the
different layers that the model will try to optimize during each
training run.

\begin{Shaded}
\begin{Highlighting}[]
\NormalTok{model1<-}\StringTok{ }\KeywordTok{keras_model_sequential}\NormalTok{() }\OperatorTok\StringTok{ }
\StringTok{  }\KeywordTok{layer_conv_2d}\NormalTok{(}\DataTypeTok{filters =} \DecValTok{32}\NormalTok{, }\DataTypeTok{kernel_size =} \KeywordTok{c}\NormalTok{(}\DecValTok{3}\NormalTok{,}\DecValTok{3}\NormalTok{), }\DataTypeTok{activation =} \StringTok{"relu"}\NormalTok{, }
                \DataTypeTok{input_shape =} \KeywordTok{c}\NormalTok{(}\DecValTok{64}\NormalTok{,}\DecValTok{64}\NormalTok{,}\DecValTok{1}\NormalTok{)) }\OperatorTok
\StringTok{  }\KeywordTok{layer_max_pooling_2d}\NormalTok{(}\DataTypeTok{pool_size =} \KeywordTok{c}\NormalTok{(}\DecValTok{3}\NormalTok{,}\DecValTok{3}\NormalTok{)) }\OperatorTok
\StringTok{  }\KeywordTok{layer_flatten}\NormalTok{() }\OperatorTok
\StringTok{  }\KeywordTok{layer_dropout}\NormalTok{(}\DataTypeTok{rate=}\FloatTok{0.5}\NormalTok{) }\OperatorTok\StringTok{ }
\StringTok{  }\KeywordTok{layer_dense}\NormalTok{(}\DecValTok{1}\NormalTok{, }\DataTypeTok{activation=}\StringTok{"softmax"}\NormalTok{)}
\end{Highlighting}
\end{Shaded}

\begin{Shaded}
\begin{Highlighting}[]
\NormalTok{model1 }\OperatorTok\StringTok{ }\KeywordTok{compile}\NormalTok{(}
  \DataTypeTok{optimizer =} \StringTok{"adam"}\NormalTok{,}
  \DataTypeTok{loss =} \StringTok{"binary_crossentropy"}\NormalTok{,}
  \DataTypeTok{metrics =} \StringTok{"accuracy"}
\NormalTok{)}
\KeywordTok{summary}\NormalTok{(model1)}
\end{Highlighting}
\end{Shaded}

\begin{verbatim}
## Model: "sequential"
## ________________________________________________________________________________
## Layer (type)                        Output Shape                    Param #     
## ================================================================================
## conv2d (Conv2D)                     (None, 62, 62, 32)              320         
## ________________________________________________________________________________
## max_pooling2d (MaxPooling2D)        (None, 20, 20, 32)              0           
## ________________________________________________________________________________
## flatten (Flatten)                   (None, 12800)                   0           
## ________________________________________________________________________________
## dropout (Dropout)                   (None, 12800)                   0           
## ________________________________________________________________________________
## dense (Dense)                       (None, 1)                       12801       
## ================================================================================
## Total params: 13,121
## Trainable params: 13,121
## Non-trainable params: 0
## ________________________________________________________________________________
\end{verbatim}

After training this model we can see that although it starts at around
being 75\% accurate on both the training and validation sets, that even
after 5 epochs that the model does not do any better at being able to
discern between whether or not an image is normal or pneumonia than just
random guessing. This suggests that either the model does not have
enough layers to it and is not able to extract distinguishable features
from the images or that there were not enough epochs to train the model.
It is also worth noting that this model took 2 and a half minutes to
complete its training, with roughly 30 seconds per epoch. This long run
time is due to the large number of trainable parameters in the model.

\hypertarget{model-2}{%
\subsection{Model 2}\label{model-2}}

~~~~~Since model 1 did not do any better than one would do than just
randomly guessing, I decided to try to improve upon the first model by
adding a second convolution layer after the first pooling layer, as well
as a second dense layer at the end of the model. I also decided to
increase the number of epochs from 5 to 10 to see if allowing the model
a longer period of time to train itself would help improve the accuracy
of the model at all. My hopes were that adding these new layers would
help the model discover features in the images that it wasnt able to
detect with the first model and adding more epochs would give the model
more time to find these features. Figure 7 shows how the new model is
constructed. We can see that by adding these two new layers to the model
that there are now 34,659 parameters that the model is going to try to
maximize during its training process.

Even after adding a second convolution layer and a second dense layer at
the end, we can see that this model again does not do any better than
the previous model. One can see that the accuracy stays right around
75\% for both the training and validation sets, and that even after
increasing the complexity of the model slightly and increasing the
number of epochs that the model is still not any better than guessing
Pneumonia every time. This model also took roughly 5 minutes to run with
every epoch taking 30 seconds to run.

\hypertarget{model-3}{%
\subsection{Model 3}\label{model-3}}

~~~~~Since my second model still did not do much better than someone
randomly guessing, for my third model I decided to increase the
complexity of the model substantially. Since the previous models did not
appear to be able to detect any distinguishable features from the
images, by increasing the number of layers in the model I believe it
should help the model finally be able to detect the underlying features.
For this model I decided to have 4 pairs of convolution, pooling, and
dropout layers, followed by a flattening layer and then two dense
layers. As we can see from Figure 9, this model now has 98,017 trainable
parameters that the model will try to optimize.

As we can see from this output, the model is finally able to discern
between images that contain pneumonia and normal images. As we can, over
the course of 10 epochs the model is able to increase it accuracy from
under 75\% to close to 90\% by the 10th epoch. From the graph we can
also see that the validation accuracy score is always slightly under the
training accuracy score, which suggests that the model might be slightly
over-fitting the data, but because I already included 4 dropout layer in
the model and the scores are still relatively close, this is not
something to worry about too much. Now that I have found a model that is
able to achieve close to 90\% accuracy, I am going to move on and try
this model on the testing set. This model also took close to 5 minutes
to run even though the number of trainable parameters increased
substantially.

\hypertarget{results}{%
\subsection{Results}\label{results}}

~~~~~After running model 3 on the testing set, the model was able to
achieve an accuracy score of 88.94\% and a loss score of 0.325 (which
would ideally be at 0 if there was perfect accuracy). The testing set
contained 234 normal images and 390 pneumonia images so if one were to
guess pneumonia every time they would be correct 62.5\% of the time.
While in an ideal world the model would be able to achieve a perfect
100\% accuracy and 0 loss, being able to train a model that achieves an
accuracy of 89\% on a task as challenging as being able to identify
whether or not someone has pneumonia is a very promising result.

When we break down the accuracy score further and look into how the
model did at identifying the two different classes one can begin to look
into areas where the model does very well and where the model struggles.
By looking at the confusion matrix in Figure 11, we can see that the
model does an extremely good job of identifying when the image is of
someone who has pneumonia, being able to predict is correctly 96.4\% of
the time. We can also see that the model is only able to correctly
predict whether someone is normal at a 76.5\% of the time. What this
output tells us is that the model is more likely to identify a patient
as having pneumonia when they have do not have it, rather than the
opposite scenario of telling a patient they do not have pneumonia when
they in fact have it which is probably the more dangerous scenario. When
thinking about why the model does such a better job at identifying
patients who have pneumonia, it comes back to the type of data we used
to train the model. Since the training and validation data contained 3x
as many pictures of pneumonia, it makes sense that the model does a much
better job of being able to recognize when someone has pneumonia. So
while being able to achieve 89\% accuracy with a convolutional neural
network is very promising, these results show that there is future work
that can be done that would be able to improve the model's accuracy even
more.

\hypertarget{conclusion-and-future-work}{%
\section{Conclusion and Future Work}\label{conclusion-and-future-work}}

\begin{itemize}
\tightlist
\item
  working with google cloud platform environment, not having to store
  the data on my computer at all -model could have been optimized
\item
  further application areas, such as lesions and other things that can
  be detected in ct scans (enhancing images
  themself)\url{https://healthitanalytics.com/news/deep-learning-model-can-enhance-standard-ct-scan-technology}
\end{itemize}

`

\bibliographystyle{agsm}
\bibliography{bibliography.bib}

\end{document}
